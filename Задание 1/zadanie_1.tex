\documentclass[12pt,a4paper]{article}
\usepackage[utf8]{inputenc}
\usepackage[russian]{babel}
\usepackage[OT1]{fontenc}
\usepackage{amsmath}
\usepackage{amsfonts}
\usepackage{amssymb}
\usepackage{color}
\begin{document}
\begin{flushleft}
\definecolor{onE}{RGB}{243,95,108}
\definecolor{twO}{RGB}{95,108,243}


\textrm{\colorbox{onE}{\it EquationEditor} --- $\overline{\mbox {это средство визуального редактирования}}$ ,  \underline{предо"=} \underline{ставляющее набор стандартных математических конструкций} , которые вы можете заполнять числами, специальными символами и другими структурными частями формул. Чтобы отредактировать одну из формул, созданных с помощью {\it EquationEditor}, достаточно дважды щелкнуть по ней или, выделив формулу, выбрать команду Правка ? Объект Equation, а из появившегося подменю выбрать пункт Изменить. Это приведет к запуску {\it EquationEditor} и вставке в него выбранной \textcolor{twO}{формулы для редактирования.}}

\end{flushleft}

\textsf{Примеры {\Huge математических формул}, приведены в этой главе, изображены так, как вы видите их на экране.}

\begin{flushright}


\texttt{{\it EquationEditor} предоставляет немало мощных средств для настройки внешнего вида и процесса набора формул. В то же время стандартные настройки и стили {\it {\bfseries EquationEditor}} \fbox{подходят для большинства математических, научных и деловых  работ.}}


\end{flushright}

\itshape{Создание формулы напоминает сборку трехмерной головоломки: соединяя составные части по одной, вы стремитесь создать завершенную форму, к примеру, шар или куб. Если {\bfseriesодна из составных частей} установлена неверно, то конечного результата вам достигнуть не удастся.}


\begin{itemize}
\item Метод Ньютона.
\item Метод Лагранжа.
\item Метод линеаризации.
\end{itemize}
\begin{itemize}
\item[$+$] Доказана сходимость.
\item[$-$] Нет оценки скорости.
\item[(?)] Приводится текст программы.
\end{itemize}




\end{document}