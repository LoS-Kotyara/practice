\documentclass[12pt,a4paper]{article}
\usepackage[utf8x]{inputenc}
\usepackage{ucs}
\usepackage[russian]{babel}
\usepackage[OT1]{fontenc}
\usepackage{amsmath}
\usepackage{amsfonts}
\usepackage{amssymb}
\author{Давиденко Алексей}

\begin {document}
{\large {\bf Связанные понятия:}}
\begin{enumerate}
\item Образом подмножества $  M  \subset  L_k$ относительно линейного отображения {\it A} называется множество $AM=\{A_x\colon x \in M\}$
\item \textit{Ядром}  линейного отображения $\{f\colon A \to B \}$ называется подмножество {\it A} , которое отображается в нуль: 

\medskip
$Ker f = \{x \in A \bigm | f(x)=0\}$ 


Ядро линейного отображения образует подпространство в линейном пространстве {\it A}.
\item \textit{Образом} линейного отображения  называется следующее подмножество {\it B}:

\medskip
Im f =  $\{f(x) \in B \bigm | x \in A\}$


Образ линейного отображения образует подпространство в линейном пространстве {\it B}.

\medskip
\item Отображение $f\colon A \times B \to C $ прямого произведения линейных пространств {\it A} и {\it B} в линейное пространство C называется билинейным, если оно линейно по обоим своим аргументам. Отображение прямого произведения большего числа линейных пространств $f\colon A_1 \times\dots\times A_n \to B$ называется полилинейным, если оно линейно по всем своим аргументам.
\item Оператор $\widetilde{L}$ называется \textit{линейным неоднородным} (или \textit{аффинным}), если он имеет вид

$\widetilde{L} = L+u$

где $L$ --- линейный оператор, а $u$ --- вектор.
\item Пусть $A\colon L_k \to L_k$. Подпространство $M$  $\subset  L_k$ называется \textit{инвариантным} относительно линейного отображения, если $\forall x \in M , Ax \in M$.

Критерий инвариантности. Пусть $ M \subset X$ --- подпространство,такое что $X$ разлагается в прямую сумму: $X= M \otimes N$. Тогда M инвариантно относительно линейного отображения $A$ тогда и только тогда, когда $P_M A P_M = AP_M$ , где $ P_M$ - проектор на подпространство $M$.
\item {\bf Фактор-операторы}. Пусть $A \colon L_k \to L_k$ --- линейный оператор и пусть M --- некоторое инвариантное относительно этого оператора подпространство. Образуем фактор-пространство $ L_k\bigl/^M\sim$ по подпространству $M$ Тогда {\bf фактор-оператором} называется оператор $A^+$ действующий на $ L_k\bigl/^M\sim$ по правилу: $\forall x^+ \in  L_k\bigl/^M\sim, A^+ x^+ = [Ax$], где $[Ax]$ --- класс из фактор-пространства, содержащий $Ax$.
\end{enumerate}
{\large {\bf Примеры линейных однородных операторов:}}

\begin{itemize}
\item оператор дифференцирования: $L\{x(\cdot)\} = y(t) = \frac {dx(t)} {dt}$;
\item оператор интегрирования: $y(t) = \int\limits_0^t x(\tau)d\tau$;
\item оператор умножения на определённую функцию $\varphi(t)\colon y(t) =  \varphi(t)\colon x(t)$;
\item оператор интегрирования с заданным «весом» $\varphi(t)\colon y(t) =  \int\limits_0^t x(\tau)\varphi(\tau)d\tau$;
\item оператор взятия значения функции f в конкретной точке $x_0: L\{f\} = ~f(x_0)$;
\item оператор умножения вектора на матрицу: $ b = Ax$;
\item оператор поворота вектора.
\end{itemize}
{\large{\bf Примеры линейных неоднородных операторов:}}


\begin{itemize}
\item Любое аффинное проеобразование;
\item $y(t) = \frac {dx(t)} {dt} + \varphi(t)$;
\item $y(t) = \int\limits_0^t x(\tau)d\tau+\varphi(t)$;
\item $y(t) = \varphi_1(t)x(t)+\varphi_2(t)$;
\end{itemize}
где $ \varphi(t), \varphi_1(t), \varphi_2(t)$ --- вполне определённые функции, а x(t) --- преобразуемая оператором функция.
\end{document}