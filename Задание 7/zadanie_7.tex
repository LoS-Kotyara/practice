\documentclass[12pt,letterpaper]{extreport}
\usepackage[14pt]{extsizes}
\usepackage{amsmath}
\usepackage{amsfonts}
\usepackage{amssymb}
\usepackage{graphicx}
\usepackage{geometry}
\usepackage{wrapfig}
%\usepackage{boxedminipage}
\geometry{a4paper,tmargin=1.5cm,bmargin=1.25cm,lmargin=3cm,rmargin=1.5cm}
\usepackage[T2A]{fontenc}
\usepackage[utf8]{inputenc}
\usepackage[russian]{babel}
\usepackage{textcase}
\renewcommand{\rmdefault}{ftm} % Times New Roman
\frenchspacing
\usepackage{tocloft}
\setlength{\cftbeforetoctitleskip}{-1em}
\setlength{\cftaftertoctitleskip}{1em}

\makeatletter
\renewcommand{\l@part}{\@dottedtocline{1}{0.5em}{1.2em}}
\makeatother

\makeatletter
\renewcommand{\l@chapter}{\@dottedtocline{1}{0.5em}{1.2em}}
\makeatother

\usepackage{epigraph}




\usepackage{fancyhdr}
\pagestyle{fancy}
\fancyhf{} 
\fancyfoot[R]{\thepage} 
\renewcommand{\footrulewidth}{0pt} 
\renewcommand{\headrulewidth}{0pt}
\fancypagestyle{plain}{ 
    \fancyhf{}
    \rfoot{\thepage}}
\usepackage{indentfirst}
\setlength{\parindent}{6ex}

\usepackage[nodisplayskipstretch]{setspace} 
\setstretch{1.1}
\onehalfspacing

\makeatletter
\newcommand\appendix@chapter[1]{%
  \renewcommand{\@makeschapterhead}[1]{\@makechapterhead{#1}}%
  \renewcommand{\thechapter}{\Asbuk{chapter}}%
  \refstepcounter{chapter}%
  \orig@chapter*{\appendixname~\thechapter~#1}%
  \addcontentsline{toc}{chapter}{\appendixname~\thechapter~~#1}%
}
\let\orig@chapter\chapter
\g@addto@macro\appendix{\let\chapter\appendix@chapter}
\makeatother



\author{Давиденко Алексей}
\begin{document}
\begin{titlepage}
\begin{center}
Министерство образования и науки Российской Федерации\\
ФЕДЕРАЛЬНОЕ ГОСУДАРСТВЕННОЕ БЮДЖЕТНОЕ\\
ОБРАЗОВАТЕЛЬНОЕ УЧРЕЖДЕНИЕ ВЫСШЕГО ОБРАЗОВАНИЯ\\
«САРАТОВСКИЙ НАЦИОНАЛЬНЫЙ ИССЛЕДОВАТЕЛЬСКИЙ ГОСУДАРСТВЕННЫЙ УНИВЕРСИТЕТ\\
ИМЕНИ Н.Г. ЧЕРНЫШЕВСКОГО»\\
\end{center}
\bigskip

\begin{flushright}
\begin{minipage}{0.5\textwidth}
\begin{flushleft}

{\small УТВЕРЖДАЮ}\\
\vspace*{.1cm}
Зав. кафедрой,\\
$\underset{\text{ уч. ст., уч. зв.}}{\text{\underline{к. ф.-м.н., доцент\hspace*{4.485cm}}}}$
\\ \vspace*{0.5cm}
\underline{\hspace*{1\textwidth}}
$\underset{\text{подпись, дата}}{\underline{\hspace*{4.25cm}}}$\hfill
$\underset{\text{иниц., фамилия}}{\text{\underline{Л.\,Б.~Тяпаев \hspace*{0.58cm}}}}$
\end{flushleft}
\end{minipage}
\end{flushright}

\vfill

\centerline{\bf  ОТЧЁТ О ПРАКТИКЕ}
\vfill
{\raggedright Студента 1 курса факультета \underline{КНиИТ}, направление 09.03.01 «Информатика и вычислительная техника»}
\begin{center}
$\underset{\text{фамилия, ямя, отчество}}{\text{\underline{Давиденко Алексея Алексеевича}}}$\\
$\underset{\text{вид практики}}{\text{\underline{учебная (ознакомительная)}}}$\\
$\underset{\text{кафедра}}{\text{\underline{Дискретной математики и информационных технологий}}}$
\end{center}

\begin{flushleft}
курс \underline{1}
\\семестр \underline{2}
\\продолжительность $\underset{\text{кол. недель, сроки  практики}}{\text{\underline{2 недели, с 29.06.2018 г. по 12.07.2018 г.}}}$
\end{flushleft}


\vfill
\leftline{Руководитель практики}
\vfill
\rightline{$\underset{\text{должность,  уч. ст., уч. зв.}}{\text{\underline{ассистент кафедры ДМиИТ~}}}$
\hfill$\underset{\text{личная подпись, дата}}{\underline{\raisebox{-2.75pt}{\hspace*{4.25cm}}}}$
\hfill$\underset{\text{инициалы, фамилия}}{\text{\underline{В.\,А.~Поздняков}}}$}

\vfill





\end{titlepage}
\begin{enumerate}
\item БИОЛОГИЧЕСКИЙ ФАКУЛЬТЕТ

\begin{enumerate}
\item Биология (Бакалавр)
\item Биология (Магистр)
\item Биологические науки	Аспирант
\item Педагогическое образование-(п)биол	Бакалавр
\item Педагогическое образование-биол	Магистр
\end{enumerate}

\item ГЕОГРАФИЧЕСКИЙ ФАКУЛЬТЕТ

\begin{enumerate}
\item Экология и природопользование	Бакалавр
\item География	Бакалавр
\item Картография и геоинформатика	Бакалавр
\item Прикладная гидрометеорология	Бакалавр
\item Экология и природопользование	Магистр
\item География	Магистр
\item Прикладная гидрометеорология	Магистр
\item Науки о земле	Аспирант
\end{enumerate}

\item ГЕОЛОГИЧЕСКИЙ ФАКУЛЬТЕТ
\begin{enumerate}
\item Геология	Бакалавр
\item Геология	Магистр
\item Нефтегазовое дело	Бакалавр
\item Прикладная геология	Специалист
\end{enumerate}

\item ИНСТИТУТ ДОПОЛНИТЕЛЬНОГО ПРОФЕССИОНАЛЬНОГО ОБРАЗОВАНИЯ
\begin{enumerate}
\item Менеджмент	Бакалавр
\item Менеджмент	Магистр
\item Психолого-педагогическое образование(спо)	Бакалавр
\end{enumerate}

\item ИНСТИТУТ ИСКУССТВ
\begin{enumerate}
\item Педагогическое образование-(п)муз	Бакалавр
\item Педагогическое образование-иск	Магистр
\item Народная художественная культура	Магистр
\item История искусств	Бакалавр
\item Хореографическое искусство	Бакалавр
\item Музыкальное искусство эстрады	Бакалавр
\end{enumerate}

\item ИНСТИТУТ ИСТОРИИ И МЕЖДУНАРОДНЫХ ОТНОШЕНИЙ
\begin{enumerate}
\item Международные отношения	Бакалавр
\item Международные отношения	Магистр
\item Сервис	Бакалавр
\item Туризм	Бакалавр
\item Сервис	Магистр
\item Туризм	Магистр
\item Педагогическое образование-история	Бакалавр
\item Педагогическое образование-ист	Магистр
\item История	Бакалавр
\item История	Магистр
\item Исторические науки и археология	Аспирант
\end{enumerate}

\item ИНСТИТУТ ФИЗИЧЕСКОЙ КУЛЬТУРЫ И СПОРТА
\begin{enumerate}
\item Педагогическое образование-(п)физ к	Бакалавр
\item Педагогическое образование-физра	Магистр
\item Физическая культура	Магистр
\item Физическая культура	Бакалавр
\end{enumerate}

\item ИНСТИТУТ ФИЛОЛОГИИ И ЖУРНАЛИСТИКИ
\begin{enumerate}
\item Журналистика	Бакалавр
\item Журналистика	Магистр
\item Педагогическое образование-(п)фил обр	Бакалавр
\item Педагогическое образование-(п)фр яз	Бакалавр
\item Педагогическое образование-фил	Магистр
\item Филология-отеч	Бакалавр
\item Филология-зар(англ)	Бакалавр
\item Филология-зар(нем)	Бакалавр
\item Фундаментальная и прикладная лингвистика	Бакалавр
\item Филология-ром гер	Магистр
\item Филология-рус слов	Магистр
\item Филология-рус яз	Магистр
\item Филология-теор яз	Магистр
\item Языкознание и литературоведение	Аспирант
\end{enumerate}

\item ИНСТИТУТ ХИМИИ
\begin{enumerate}
\item Химия	Бакалавр
\item Химия	Магистр
\item Химические науки	Аспирант
\item Химическая технология	Бакалавр
\item Химическая технология	Магистр
\item Техносферная безопасность	Бакалавр
\item Педагогическое образование-хим	Бакалавр
\item Педагогическое образование-хим	Магистр
\end{enumerate}

\item МЕХАНИКО-МАТЕМАТИЧЕСКИЙ ФАКУЛЬТЕТ
\begin{enumerate}
\item Механика и математическое моделирование	Бакалавр
\item Прикладная математика и информатика	Бакалавр
\item Прикладная математика и информатика	Магистр
\item Математика и механика	Аспирант
\item Математика и компьютерные науки	Бакалавр
\item Математика и компьютерные науки	Магистр
\item Прикладная информатика	Бакалавр
\item Прикладная информатика	Магистр
\item Бизнес-информатика	Бакалавр
\item Педагогическое образование(п)мехмат	Бакалавр
\end{enumerate}

\item СОЦИОЛОГИЧЕСКИЙ ФАКУЛЬТЕТ
\begin{enumerate}
\item Прикладная информатика	Бакалавр
\item Государственное и муниципальное управление	Бакалавр
\item Организация работы с молодежью	Бакалавр
\item Социальная работа	Бакалавр
\item Социология	Бакалавр
\item Социология	Магистр
\item Социологические науки	Аспирант
\end{enumerate}

\item ФАКУЛЬТЕТ ИНОСТРАННЫХ ЯЗЫКОВ И ЛИНГВОДИДАКТИКИ
\begin{enumerate}
\item Педагогическое образование-анг	Бакалавр
\item Педагогическое образование-нем	Бакалавр
\item Педагогическое образование-иняз	Магистр
\end{enumerate}
\item ФАКУЛЬТЕТ КНИИТ
\begin{enumerate}
\item Фундаментальная информатика и информационные технологии	Бакалавр
\item Математическое обеспечение и администрирование информационных систем	Бакалавр
\item Математическое обеспечение и администрирование информационных систем	Магистр
\item Компьютерные и информационные науки	Аспирант
\item Информатика и вычислительная техника	Бакалавр
\item Программная инженерия	Бакалавр
\item Информатика и вычислительная техника	Магистр
\item Информатика и вычислительная техника	Аспирант
\item Компьютерная безопасность	Специалист
\item Системный анализ и управление	Бакалавр
\item Педагогическое образование-(п)книит	Бакалавр
\item Педагогическое образование-книит	Магистр
\end{enumerate}
\item ФАКУЛЬТЕТ НЕЛИНЕЙНЫХ ПРОЦЕССОВ
\begin{enumerate}
\item Прикладные математика и физика	Бакалавр
\item Радиофизика	Бакалавр
\item Прикладные математика и физика	Магистр
\item Радиофизика-нел	Магистр
\item Информационные системы и технологии	Бакалавр
\item Информационные системы и технологии
\end{enumerate}
\item ФАКУЛЬТЕТ НИБМТ
\begin{enumerate}

\item Физика	Бакалавр
\item Физика-нано	Магистр
\item Электроника и наноэлектроника	Бакалавр
\item Электроника и наноэлектроника	Магистр
\item Электроника, радиотехника и системы связи	Аспирант
\item Биотехнические системы и технологии	Бакалавр
\item Биотехнические системы и технологии	Магистр
\item Материаловедение и технологии материалов	Бакалавр
\item Материаловедение и технологии материалов	Магистр
\item Инноватика	Бакалавр
\item Управление качеством	Бакалавр
\item Управление качеством	Магистр
\end{enumerate}
\item ФАКУЛЬТЕТ ПСИХОЛОГИИ
\begin{enumerate}
\item Психология	Бакалавр
\item Психология	Магистр
\item Психологические науки	Аспирант
\end{enumerate}
\item ФАКУЛЬТЕТ ПСИХОЛОГО-ПЕДАГОГИЧЕСКОГО И СПЕЦИАЛЬНОГО ОБРАЗОВАН
\begin{enumerate}
\item Педагогическое образование-дош обр	Бакалавр
\item Педагогическое образование-нач обр	Бакалавр
\item Педагогическое образование-техн	Бакалавр
\item Психолого-педагогическое образование-псих и соц пед	Бакалавр
\item Психолого-педагогическое образование-псих обр	Бакалавр
\item Специальное (дефектологическое) образование-лог	Бакалавр
\item Специальное (дефектологическое) образование-олиг	Бакалавр
\item Специальное (дефектологическое) образование-спец псих	Бакалавр
\item Специальное (дефектологическое) образование-сурдопед	Бакалавр
\item Педагогическое образование-нач обр	Магистр
\item Педагогическое образование-техн обр	Магистр
\item Психолого-педагогическое образование-диаг и кор	Магистр
\item Психолого-педагогическое образование-соц пед	Магистр
\item Специальное (дефектологическое) образование-деф	Магистр
\item Специальное (дефектологическое) образование-псих пед сопр	Магистр
\item Специальное (дефектологическое) образование-спец псих	Магистр
\item Образование и педагогические науки	Аспирант
\end{enumerate}
\item ФИЗИЧЕСКИЙ ФАКУЛЬТЕТ
\begin{enumerate}
\item Физика	Бакалавр
\item Радиофизика	Бакалавр
\item Физика-физ	Магистр
\item Радиофизика-физ	Магистр
\item Физика и астрономия	Аспирант
\item Инфокоммуникационные технологии и системы связи	Бакалавр
\item Конструирование и технология электронных средств	Бакалавр
\item Биотехнические системы и технологии	Бакалавр
\item Педагогическое образование-физ	Бакалавр
\item Педагогическое образование-физ	Магистр
\end{enumerate}
\item ФИЛОСОФСКИЙ ФАКУЛЬТЕТ
\begin{enumerate}
\item Педагогическое образование-филос	Бакалавр
\item Педагогическое образование-филос	Магистр
\item Философия	Бакалавр
\item Религиоведение	Бакалавр
\item Философия-ист фил	Магистр
\item Философия-соц фил	Магистр
\item Религиоведение	Магистр
\item Философия, этика и религиоведение	Аспирант
\item Теология	Бакалавр
\item Теология	Магистр
\item Теология	Аспирант
\item Культурология	Бакалавр
\item Культурология	Магистр
\end{enumerate}
\item ЭКОНОМИЧЕСКИЙ ФАКУЛЬТЕТ
\begin{enumerate}
\item Экономика	Бакалавр
\item Менеджмент	Бакалавр
\item Управление персоналом	Бакалавр
\item Экономика-эк ин раз	Магистр
\item Экономика-фин план	Магистр
\item Экономика	Аспирант
\end{enumerate}
\item ЮРИДИЧЕСКИЙ ФАКУЛЬТЕТ
\begin{enumerate}
\item Таможенное дело	Специалист
\item Юриспруденция	Бакалавр
\item Юриспруденция	Магистр
\item Судебная экспертиза	Специалист
\item Юриспруденция	Аспирант
\item Политология	Бакалавр
\item Политология	Магистр
\item Реклама и связи с общественностью	Бакалавр
\end{enumerate}
\end{enumerate}

\addcontentsline{toc}{chapter}{Литература}
\begin{thebibliography}{99}
\bibitem{1}
Роберт Лав. Ядро Linux: описание процесса разработки = Linux Kernel Development. — 3-е изд. — М.: Вильямс, 2012. — 496 с. — ISBN 978-5-8459-1779-9.

\bibitem{2}
Дмитрий Кетов. Linux. Внутреннее устройство / Анна Кузьмина. — СПб.: БХВ, 2017. — 320 с. — ISBN 978-5-9775-3580-9.

\bibitem{3}
Костромин В. А. Свободная система для свободных людей (обзор истории операционной системы Linux) (рус.). rus-linux.net (10 мая 2005).

\bibitem{4}
Нейл Ботвик, Энди Ченнел. Many happy return()s! Долгих лет тебе, Linux! (рус.) // Linux Format : журнал. — СПб., 2006. — Декабрь (№ 12 (86)). — С. 22-33.

\bibitem{5}
Маянк Шарма. Рождение ядра Linux // Linux Format. — 2016. — Октябрь (№ 10 (215)). — С. 24-31.


\end{thebibliography}

\end{document}