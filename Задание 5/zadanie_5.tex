\documentclass[12pt,A4, twoside]{article}
\usepackage[14pt]{extsizes}
\usepackage[utf8]{inputenc}
\usepackage[russian]{babel}
\usepackage{amsmath}
\usepackage{amsfonts}
\usepackage{amssymb}
\usepackage{graphicx}
\usepackage{color}
\usepackage{geometry}
\usepackage{wrapfig}
\geometry{a4paper,tmargin=2cm,bmargin=2cm,lmargin=3cm,rmargin=1.5cm}


\definecolor{one}{RGB}{90,90,90}
\definecolor{two}{RGB}{250,250,250}


\begin{document}
\pagecolor{two}
\pagestyle{empty}

\begin{flushright}
\begin{Huge}
\textsc{День защитника Отечества}
\end{Huge}
\begin{LARGE}
\\ \bigskip \textsc{\textcolor{one}{23 февраля}}
\end{LARGE}
\end{flushright}


\begin{wrapfigure}[1]{l}{0.45\textwidth}
	\centering
	\includegraphics[scale=0.75]{23_feb}
	
\end{wrapfigure}

\textsl{
	\begin{flushright}
		Стоим мы на посту, повзводно\\
		и поротно.\\
		Бессмертны, как огонь. Спокойны, как\\
		гранит.\\
		Мы — армия страны. Мы — армия\\
		народа.\\
		Великий подвиг наш история хранит\\ \medskip
		\em{\textrm{Музыка Г.~Мовсесяна,\\слова Р.~Рождественского}}
	\end{flushright}
}

\bigskip\bigskip

\par 23 февраля отмечается один из дней воинской славы России — \textbf{День защитника Отечества}.
Эта дата была установлена Федеральным законом «О днях воинской славы и памятных датах России»,
принятым Государственной думой и подписанным президентом РФ Б.Ельциным 13 марта 1995 года.

\par Принято было считать, что 23 февраля 1918 года отряды Красной гвардии одержали свои первые
победы под Псковом и Нарвой над регулярными войсками кайзеровской Германии. Вот эти первые победы
и стали «днем рождения Красной Армии».

\par В 1922 году эта дата была официально объявлена \textbf{Днем Красной Армии}. Позднее 23 февраля
ежегодно отмечался в СССР как всенародный праздник — День Советской Армии и Военно-Морского Флота.

\par 10 февраля 1995 года Государственная Дума России приняла Федеральный закон «О днях воинской славы
(победных днях) России», в котором 23 февраля имеет следующее название: «День победы Красной армии над
кайзеровскими войсками Германии (1918 год) — День защитников Отечества».

\par Федеральным законом № 48-ФЗ «О внесении изменения в статью 1 Федерального закона «О днях воинской
славы и памятных датах России», принятым 15 апреля 2006 года, было установлено, что «Согласно внесенным
изменениям день воинской славы России 23 февраля переименован в \textbf{День защитника Отечества}. Он
является официальным выходным днем. И, независимо от названия, в этот день всегда чествовали настоящих
мужчин — защитников своей Родины.

\par Сегодня для некоторых людей праздник 23 февраля остался днем мужчин, которые служат в армии или в
каких-либо силовых структурах. Тем не менее, большинство граждан России и стран бывшего СССР склонны
рассматривать День защитника Отечества не столько, как годовщину победы или День Рождения Красной Армии,
сколько, как День настоящих мужчин. Защитников в самом широком смысле этого слова. И для большинства
наших сограждан это важная и значимая дата.

\par Необходимо также отметить, что в этот день поздравляют не только мужчин, а еще и женщин — ветеранов
Великой Отечественной войны, женщин-военнослужащих. Среди традиций праздника, сохранившихся и сегодня, —
чествование ветеранов, возложение цветов к памятным местам, в частности в Москве — это торжественное
возложение венков к Могиле Неизвестного Солдата у стен Кремля первыми лицами государства. А также
проведение праздничных концертов и патриотических акций, организация салютов во многих городах России.

\par Кстати, до 1917 года \textbf{традиционно днем Русской армии являлся праздник} 6 мая ~--— День святого Георгия Победоносца, считающегося покровителем русских воинов.

\par Вместе с Россией этот праздник традиционно отмечают в Беларуси и Кыргызстане.

\end{document}