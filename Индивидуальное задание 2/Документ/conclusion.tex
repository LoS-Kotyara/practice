\conclusion
То, что зарождалось как программа для подключения к университетскому компьютеру, превратилось в самый грандиозный проект мира свободного программного обеспечения. Сегодня по данным Евросоюза, стоимость разработки ядра Linux с нуля при коммерческом подходе, составляет более одного миллиарда евро. Модель коллективной разработки СПО доказала свою жизнеспособность. Для многих оказалось открытием, возможность достойной конкуренции разработки кучки энтузиастов против продуктов транснациональных корпораций с многомиллиардными оборотами. Linux в очередной раз, доказал, что деньги в этом мире не главное, и добрая воля человека способна на великие свершения.