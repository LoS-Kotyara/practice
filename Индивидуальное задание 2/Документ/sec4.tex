\section{РАСПРОСТРАНЕНИЕ LINUX}
\par Широкое распространение операционной системы Linux началось со времени выхода стабильной версии ядра версии 2.2 в январе 1999 года. На нее обратили внимание производители серверных приложений, баз данных, Web-серверов, а также приложений для всякого рода защиты ПК. Произошло это во многом благодаря широкому распространению веб-сервера Apache. На сегодняшний день порядка 65\% web-серверов работают на ОС Linux, по данным TOP500, Linux используется на 91 \% самых мощных суперкомпьютеров планеты и на подавляющее большинстве компьютеров обслуживающих систему доменных имён DNS (без которой не возможно функционирование сегодняшней сети интернет). Инфраструктура самой популярной поисковой системы Google.com и сайта wikipedia.org (шестого в мировом рейтинге), строится на базе множества серверов с Linux. Можно сказать, что на серверах Linux чувствует себя уверенно и пришел на них на долго.
\par Начиная с 1998 года, о поддержке, распространении и продаже Linux начали заявлять крупнейшие IT-компании - гиганты компьютерного рынка. В их число входят: Sun, IBM, Oracle, Hewlett-Packard, Novell. Эти компании начали устанавливать Linux на свои сервера, адаптировать под него свои программные продукты. По-другому взглянули на свободное программное обеспечение и правительства стран, администрации городов. Зачастую они стали отказываться от продуктов корпорации Microsoft в пользу Linux и СПО, экономя при этом, огромные деньги. В число таких стран входят Германия, Франция, Англия, Япония. Порой целые города, муниципальные службы и министерства в них переходили на СПО.
\par Так же большую популярность, благодаря своей гибкости и свободности, Linux завоевал на различных встраиваемых и мобильных устройствах. Порой мы даже не подозреваем об обилии Linux вокруг нас. Различные модемы и роутеры, терминалы и тонкие клиенты, промышленные станки и системы видеонаблюдения, коммуникаторы и смартфоны. Диапазон применения системы очень широк.
\par Несколько другая ситуация на рынке настольных систем. Там царит гегемония продуктов Microsoft. По разным оценкам, доля ОС Linux составляет порядка 1\% -5\% от общего числа. Этому есть целый рад причин. Во-первых, долгое время в Linux отсутствовали программы к которым пользователи привыкли в среде Windows. В частности это относилось к офисным пакетам, программам обработки звука, инженерными системам и играм. На данный момент ситуация гораздо лучше, но всё же не идеальна. Вторая причина - поддержка аппаратных средств. Далеко не все производители выпускают драйвера для ОС Linux, ввиду малочисленности их пользователей. Драйвера приходится писать энтузиастам, зачатую устройства имеют ограниченную функциональность, а то и вовсе не работают. Хотя и здесь ситуация за последнее время значительно изменилась в лучшую сторону. Сегодня поддерживается огромное количество устройств, и каждый день этот список расширяется. К тому же многие производители периферии осознали значимость Linux, и сами стали выпускать драйвера для своих продуктов. И последняя причина, это банальная привычка. Для многих Windows и компьютер стали синонимами, и освоение новой системы их пугает. Усугубляется это тем, что изначально, конфигурирование Linux, предполагает работу в командной строке, а графическая оболочка это лишь удобная надстройка для повседневной работы. Многим этот принцип кажется сложным. Не говоря о гибкости и широких возможностях командного интерфейса, можно сказать что современные дистрибутивы вроде Ubuntu предоставляют богатый инструментарий по настройке именно в графическом интерфейсе. К тому же установка этого дистрибутива на компьютер не сложнее установки Windows, т.к. один из главных принципов построения этого дистрибутива - дружелюбный для пользователя интерфейс.
\par Благодаря изменениям последних лет, число инсталляций Linux всё время растёт. Ясно что эта система имеет большое будущее. В компьютерных магазинах, зачатую, помимо Windows, можно увидеть Linux как предустановленную систему. В России идёт процесс внедрения Linux и свободного программного обеспечения в школах и государственных учреждениях.