\section{РАЗВИТИЕ LINUX}
\par После выпуска версии 1.0, ядро продолжило свое развитие в виде двух веток - стабильной (рекомендуемой к широкому использованию) и экспериментальной (тестовая версия, включающее новые возможности и активно разрабатываемое). Стабильные версии имели чётную вторую цифру в номере (например 1.0.1), а экспериментальные нечётную (например 1.1.4). После того как экспериментальная версия была достаточно обработана и годилась к использованию широкими слоями пользователей, её второй номер увеличивался на единицу и она считалась стабильной. Одновременно с этим появилась новая экспериментальная версия.
\par Разработка Linux всё время набирала обороты. Если в версии 0.1 имелось всего 8 400 строк кода, то в версии 1.0 уже 170 000. В июне 1996 система уже поддерживала множество архитектур, и многопроцессорную технологию. Дальнейшее развитие в основном было направленно на улучшение производительности, поддержке новых технологий и аппаратных средств. Вообще, именно на последний пункт, приходилась большая часть кода ядра, которая к январю 2001 года превышало число в 3 000 000 строк. Программисты стремились создавать драйвера для как можно большего количества оборудования. Порою это было не простой задачей, т.к. многие производители не считали систему заслуживающей внимания, не писали для неё драйверов и не открывали спецификации на свои устройства.
\par В это время Торвальдс уже практически отошел от прямой разработки ядра, и его основной обязанностью стало руководство процессом разработки. Он выбирал направления развития и принимал решения о включении патчей, присылаемых ему разработчиками со всего мира. Кроме того Линус распределял полномочия по руководству разработкой отдельных направлений различным участникам сообщества, сам же сосредоточился на основополагающих компонентах.
\par В 1996 году был выбран символ $\textrm{системы}^\ref{fig:linux_symbol}$. Им стал добродушный и в меру упитанный пингвинёнок Такс, отличительная особенность которого - жёлтые лапы и клюв.
\par Одной из проблем этого времени стала стандартизация. Дистрибутивов становилось всё больше, многие из них были похожи друг на друга, другие разительно отличались по многим параметрам, начиная от структуры файловой системы и системы инициализации и заканчивая используемыми библиотеками и конфигурацией ядра. Это обстоятельство имело свои негативные последствия. Разработчикам приходилось адаптировать свои программы под различные дистрибутивы, на это уходило много сил и средств. Первым проектом по стандартизации был Filesystem Standart project (FSSTND). Он стартовал в августе 1993, и стандартизировал организацию файловых систем. Позже был переименован в Filesystem Hierarchy Standard или, FHS. В мае 1998 года стартовал проект Linux Standart Base (LSB), он должен был определить набор тех компонент, которые должны присутствовать в любой "Linux-системе". Инициаторы проекта ставили цель обеспечить бинарную совместимость дистрибутивов, удовлетворяющих стандарту LSB. Велись и другие проекты по стандартизации.