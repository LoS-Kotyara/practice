\section{ИСТОКИ LINUX}
\par Справедливо считается, что Linux имеет двух $\textrm{прародителей}^\ref{fig:hist}$, на основании которых он и возник. Это операционная система $\textrm{UNIX}^\ref{fig:UNIX_symbol}$ и проект $\textrm{GNU}^\ref{fig:GNU_symbol}$. О них будет рассказано ниже.
\par Linux является Unix-подобной операционной системой, совместимой с ней. Первая система Unix была разработана в 1969г. в подразделении Bell Labs компании AT\&T. В те времена компании AT\&T было запрещено заниматься компьютерным бизнесом, поэтому операционная система Unix распространялась бесплатно и её исходные коды были открыты. Это обстоятельство способствовало распространению системы в университетской среде, и стремительному её развитию. Студенты и профессора вносили в неё улучшения, создавали для неё утилиты. Коммерческие компании разрабатывали клоны системы Unix. Система стремительно набирала популярность и была установлена на множестве компьютеров. В 1983 году был реализован стек протоколов TCP/IP, что значительно расширило её сетевые возможности. В итоге, в 80-х годах, накал борьбы между производителями Unix-ов достиг максимума. В 1983 с корпорации AT\&T был снят запрет на занятие компьютерным бизнесом. Она занялась коммерциализацией свой разработки. Были закрыты исходные коды системы, а компании использующие эти коды, подвергались патентным преследованиям. После нескольких лет таких UNIX-войн развитие Unix практически сошло на нет. И UNIX уступила место на компьютерах конкурирующим системам, в частности MS DOS и Apple Macintosh.
\par Вторым прародителем Linux, можно считать проект GNU \href{https://ru.wikipedia.org/wiki/\%D0\%A1\%D1\%82\%D0\%BE\%D0\%BB\%D0\%BB\%D0\%BC\%D0\%B0\%D0\%BD,\_\%D0\%A0\%D0\%B8\%D1\%87\%D0\%B0\%D1\%80\%D0\%B4\_\%D0\%9C\%D1\%8D\%D1\%82\%D1\%82\%D1\%8C\%D1\%8E}{Ричарда Столлмана}. Он возник в 1983 году, и его целью было создание полностью свободной операционной системы. Толчком к рождению проекта стали обстоятельства возникшие в 1982 году. Тогда Ричард Столлман работал в лаборатории искусственного интеллекта Массачусетского Технологического Института. В их лабораторию была куплена коммерческая операционная система. Условия лицензирования этой системы накладывали ограничения на распространение исходных кодов программ, и это заметно тормозило процесс разработки программного обеспечения, требовало повторной разработки уже существующих компонентов. Ричард Столлман, сам будучи очень талантливым программистом решил переломить это порочное положение вещей в программировании. 27 сентября 1983 года он объявил о начале разработки проекта GNU (GNU is Not Unix) целью которого было создание Unix-совместимой операционной системы, у которой будет ядро и все необходимые сопутствующие утилиты (редактор, оболочка, компилятор и т.д.). Так же декларировалась возможность получения исходных кодов проекта любым желающим. Все желающие приглашались к участию в проекте. Чтобы МТИ не мог навязать права собственности на детище Столлмана, он ушел из института в январе 1984. Первой программой, разработанной в рамках проекта был текстовый редактор Emacs. В 1985 году Столлман основал Free Software Foundation (FSF) - благотворительный фонд для разработки свободно распространяемого ПО. Следующим очень важным шагом Ричарда было создание лицензии GPL (General Public License). Основная идея GPL в том, что пользователь должен обладать следующим правами (свободами):
	\begin{enumerate}
		\item Правом запускать программу для любых целей;
		\item Правом изучать устройство программы и приспосабливать ее к своим потребностям, что 						предполагает доступ к исходному коду программы;
		\item Правом распространять программу, имея возможность помочь другим;
		\item Правом улучшать программу и публиковать улучшения, в пользу всего сообщества, что тоже 					предполагает доступ к исходному коду программы.
	\end{enumerate}
\par Программное обеспечение, распространяемое под этой лицензией, можно как угодно использовать, копировать, дорабатывать, модифицировать, передавать, продавать модифицированные (или немодифицированные) версии другим лицам при условии, что результат такой переработки тоже будет распространяться под лицензией GPL. Последнее условие - самое важное и определяющее в этой лицензии. Оно гарантирует, что результаты усилий разработчиков свободного ПО останутся открытыми и не станут частью какого-либо проприетарного продукта.
\par К 1990 году в рамках проекта GNU было создано большинство компонент, необходимых для функционирования свободной операционной системы. Помимо текстового редактора Emacs, Столлман создал компилятор gcc (GNU C Compiler) и отладчик gdb. Так-же были разработаны библиотека языка Си и оболочка BASH. Недоставало только самого важного - ядра. В это самое время и появилась на свет разработка финского студента Линуса Торвальдса - ядро Linux. Можно сказать, что появилось оно в нужное время. И теперь симбиоз этих двух разработок зовется GNU/Linux.
