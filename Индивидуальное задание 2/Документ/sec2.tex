\section{РОЖДЕНИЕ LINUX}
\par \href{https://ru.wikipedia.org/wiki/\%D0\%A2\%D0\%BE\%D1\%80\%D0\%B2\%D0\%B0\%D0\%BB\%D1\%8C\%D0\%B4\%D1\%81,\_\%D0\%9B\%D0\%B8\%D0\%BD\%D1\%83\%D1\%81}{Линус Бенедикт Торвальдс} родился 28 декабря 1969 года. В школе он был отличным математиком, и ещё с детства начал увлекаться компьютерами. После окончания школы, он поступил в Университет Хельсинки на отделение компьютерных наук. Тогда у него был персональный компьютер на основе процессора Intel 80386 с 4 мегабайтами ОЗУ и тактовой частотой 33 мегагерца. Под впечатлением от книги Эндрю С. Таненбаума (разработчика учебной операционной системы Minix)~"Проектирование и реализация операционных систем"\,, Линус установил на свой компьютер ОС $\textrm{Minix}^\ref{fig:minix_symbol}$. Однако, молодого студента далеко не всё устраивало в этой системе. Больше всего нареканий вызывала работа терминала с помощью которого он подключался к компьютеру университета, а через него и к глобальной сети интернет. Линус принялся писать собственный терминал. После того как терминал был готов, возникала проблема со скачиванием и загрузкой файлов. Пришлось писать драйвера для флоппи-дисковода, а следом и собственную файловую систему, так как у файловой системы Minix были проблемы с многозадачностью. Так из попытки написания терминала появился скелет будущей операционной системы. Линуса заинтересовала идея создания собственной ОС и он принялся за разработку. 25 августа 1991 года Торвальдс написал e-mail в список рассылки пользователей Minix, в котором сообщал, что занимается разработкой операционной системы и просил указать пожелания и предложения от пользователей Minix. Этот день считается днём рождения Linux. А 5 октября он выпустил версию ядра 0.2 и выложил исходные коды в интернет. Многие заинтересовались этой системой. У Линуса появились помощники, работа закипела. 05.01.1992 была выпущена версия 0.12 под лицензий GPL, Linux стал достоянием всего мира. Версия 0.96 была выпущена в апреле 1992, в ней появилась возможность работы графической подсистемы X Window. И только через два года, 16.04.1994 вышел первый стабильный релиз - версия 1.0. К этому времени в рядах разработчиков уже были тысячи человек. Система динамично развивалась. В ней функционировало множество прикладного ПО. Промышленные компании и мелкие фирмы начали разрабатывать, продавать и встраивать в устройства свои версии открытой ОС. Зародились дистрибутивы Linux.
\par Дистрибутив Linux - это набор пакетов программного обеспечения, включающий базовые компоненты операционной систем (в том числе, ядро Linux), некоторую совокупность программных приложений и программу инсталляции, которая позволяет установить на компьютер пользователя операционную систему GNU/Linux и набор прикладных программ, необходимых для конкретного применения системы. Т.е. эта законченная, полнофункциональная система, уже адаптированная для применения конечным пользователем, а не только разработчиком.
\par Первые дистрибутивы Linux появились вскоре после того, как Линус Торвальдс выпустил разработанное им ядро под лицензией GPL. Отдельные программисты (и группы программистов) начали разрабатывать как программы инсталляции, так и другие прикладные программы, пользовательский интерфейс, программы управления пакетами и выпускать свои дистрибутивы.
\par Первый дистрибутив Linux был создан Оуэном Ле Бланк в феврале 1992 (Англия). В октябре 1992 появился разработанный Питером Мак-Дональдом дистрибутив Softlanding Linux System, который включал в себя такие элементы, как X Window System и поддержка TCP/IP. В конце 1992 года Патрик Фолькердинк выпустил дистрибутив который он назвал "Slackware" и который является старейшим дистрибутивом из тех, которые до сих пор активно развиваются. На основе дистрибутива Slackware германской фирмой S. U. S. E, был создан дистрибутив SuSE Linux, версия 1.0 которого вышла в 1994 году. Еще один проект по разработке дистрибутива, Debian, был начат Яном Мёрдоком 16 августа 1993 года как альтернатива коммерческим дистрибутивам Linux. Дистрибутив Red Hat, был основан в 1994 году. На основе Red Hat было создано множество других дистрибутивов.