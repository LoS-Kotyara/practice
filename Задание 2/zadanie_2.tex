\documentclass[12pt,a4paper]{article}
\usepackage[utf8]{inputenc}
\usepackage[russian]{babel}
\usepackage{amsmath}
\usepackage{amsfonts}
\usepackage{amssymb}
\usepackage{mathtools}
\begin{document}
\bfseries Линейным отображением \mdseries векторного пространства  $L_K$  над полем {\it K} в векторное пространство  (линейным оператором из $L_K$   в {\it K} ) над тем же полем {\it K}  называется отображение

$f:L_K\rightarrow M_K,$\\
удовлетворяющее {\it условию линейности}

$f(x+y)=f(x)+f(y),$
 
$f(\alpha x)= \alpha f(x)$\\
для всех $x, y \in L_K$ и $\alpha \in K$\\

Если определить операции сложения и умножения на скаляр из основного поля {\it K} как
\begin{itemize}
\item $(f+g)(x)=f(x)+g(x)~~~\forall x \in L_K$
\item $(kf)(x)=kf(x)~~~\forall x \in L_K,~\forall k \in K $
\end{itemize}
\par Множество всех линейных отображений из $L_K$ в $M_K$ превращается в векторное пространство,
которое обычно обозначается как $\mathcal{L}(L_K, M_K)$ 

Если векторные пространства $L_K$ и $M_K$ являются линейными топологическими пространствами, то есть на них определены топологии, относительно которых операции этих пространств непрерывны, то можно определить понятие ограниченного оператора: линейный оператор называется ограниченным, если он переводит ограниченные множества в ограниченные (в частности, все непрерывные операторы ограничены). В частности, в нормированных пространствах множество ограничено, если норма любого его элемента ограничена, следовательно, в этом случае оператор называется ограниченным, если существует число {\it N} такое что
$\forall x \in L_K, ||Ax||_{M_K} \leqslant N||x||_{L_K}$.
Можно показать, что в случае нормированных пространств непрерывность и ограниченность операторов эквивалентны. Наименьшая из постоянных {\it N}, удовлетворяющая указанному выше условию, называется {\bf нормой оператора}:

$||A|| = \sup\limits_{||x|| \neq 0}{\frac{||A_x ||}{||x||}} = \sup\limits_{||x||=1}{||A_x ||}.$\\

Введение нормы операторов позволяет рассматривать пространство линейных операторов как нормированное линейное пространство (можно проверить выполнение соответствующих аксиом для введенной нормы). Если пространство $M_K$  — банахово, то и пространство линейных операторов тоже банахово.

Оператор $A^{-1}$ называется обратным линейному оператору {\it A} , если выполняется соотношение:
$A^{-1}A=AA^{-1}=1$\\

Оператор $A^{-1}$ , обратный линейному оператору {\it A} , также является {\it линейным} непрерывным оператором. В случае если линейный оператор действует из банахового пространства в другое банахово пространство, то по теореме Банаха обратный оператор существует.\\

{\bfseries Унитарный оператор} — оператор, область определения и область значений которого — всё пространство, сохраняющий скалярное произведение $A_x , A_y = (x, y)$ , в частности, унитарный оператор сохраняет норму любого вектора $||A_x || = \sqrt{(A_x , A_x )} = \sqrt{(x, x)} = ||x||$ ; оператор, обратный унитарному, совпадает с сопряжённым оператором $A^{-1} = A^{*}$ ; норма унитарного оператора равна 1; в случае вещественного поля К унитарный оператор называют {\it ортогональным}; 

\end{document}