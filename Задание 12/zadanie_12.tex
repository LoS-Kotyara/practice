\documentclass[12pt,letterpaper]{extreport}
\usepackage[14pt]{extsizes}
\usepackage[utf8]{inputenc}
\usepackage[english,russian]{babel}
\usepackage{caption}
\usepackage{amsmath}
\usepackage{amsfonts}
\usepackage{amssymb}
\usepackage{graphicx}
\usepackage{geometry}
\geometry{a4paper,tmargin=1.5cm,bmargin=1.25cm,lmargin=0.5cm,rmargin=1.5cm}
\usepackage{amssymb}
\usepackage{tikz}
\usetikzlibrary{arrows,decorations.pathmorphing,shapes}
\begin{document}
\pagestyle{empty}
\tikzstyle{block} =
        [
                rectangle,
                draw,
                fill = blue!20,
                text width = 3.5cm,
                text centered,
                rounded corners,
                minimum height = 4cm
        ]
\tikzstyle{line} =
        [
                draw,
                -latex'
        ]
\tikzstyle{cloud} =
        [
                draw,
                ellipse,
                fill = red!20,
                node distance = 4cm,
                minimum height = 2cm
        ]
\begin{tikzpicture}[node distance = 4.5cm]
        \node [cloud] (start) {Начало};
        \node [block, below of = start] (phase1) {Создание объекта класса Calculating};
        \node [block, below of = phase1] (phase2) {Метода getBDay};
        \node [block, left of = phase2] (discription21){Считывание даты рождения};
       	\node [block, left of = discription21] (discription22){Проверка на правильность даты};
       	\node [block, left of = discription22] (discription23){Определение номера дня в юлианском
       	календаре};
        \node [block, below of = phase2] (phase3) {Метод getCDay};
        \node [block, left of = phase3] (discription31) {Считывание сегодняшней даты};
        \node [block, left of = discription31] (discription32){Проверка на правильность даты};
        \node [block, left of = discription32] (discription33){Определение номера дня в юлианском
       	календаре};
       	\node [block, below of = phase3] (phase4) {Нахождение разности между номером сегодняшней даты и номером даты рождения};
       	\node [block, below of = phase4] (phase5){Вывод полученного значения};
       	\node [cloud, below of = phase5] (finish){Конец};
       	
       	
       	\path [line] (start) -- (phase1);
       	\path [line] (phase1) -- (phase2);
       	\path [line] (phase2) -- (phase3);
       	\path [line] (phase3) -- (phase4);
       	\path [line] (phase4) -- (phase5);
       	\path [line] (phase5) -- (finish);
       	\path [line] (phase2) -- (discription21);
       	\path [line] (discription21) -- (discription22);
       	\path [line] (discription22) -- (discription23);
       	\path [line] (phase3) -- (discription31);
       	\path [line] (discription31) -- (discription32);
       	\path [line] (discription32) -- (discription33);
       	
\end{tikzpicture}

\end{document}