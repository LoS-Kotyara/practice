\documentclass[14pt]{beamer}
\usetheme{Dresden}
\usepackage[utf8]{inputenc}
\usepackage[russian]{babel}
\usepackage[OT1]{fontenc}
\usepackage{amsmath}
\usepackage{amsfonts}
\usepackage{amssymb}
\usepackage{graphicx}
\usepackage{wrapfig}
\usepackage{float}
\frenchspacing
\usepackage{listings}






%\author{}
\title{Высокоуровневые языки программирования}
\usecolortheme[dark,accent=green]{solarized}
\setbeamercovered{transparent} 
\setbeamertemplate{navigation symbols}{} 
\date{} 
\begin{document}

\begin{frame}[plain]
\titlepage
\end{frame}

{\section{Какие языки называются высокоуровневыми?}
\begin{frame}
\begin{block}{Высокоуровневый язык программирования "---}
язык программирования, разработанный для быстроты и удобства использования программистом. Основная черта высокоуровневых языков — это абстракция, то есть введение смысловых конструкций, кратко описывающих такие структуры данных и операции над ними, описания которых на машинном коде (или другом низкоуровневом языке программирования) очень длинны и сложны для понимания.
\end{block}
\end{frame}
}

{
\section{Примеры высокоуровневых языков}
\begin{frame}[label=menu]
\begin{alertblock}{Примеры высокоуровневых языков}
\begin{itemize}

\item \hyperlink{Python}{\texttt{Python}}
\item \hyperlink{VisualBasic}{\texttt{Visual Basic}}
\item \hyperlink{Delphi}{\texttt{Delphi}}
\item \hyperlink{Perl}{\texttt{Perl}}
\item \hyperlink{Ruby}{\texttt{Ruby}}
\item \texttt{C++}
\item \texttt{C\#}
\item \texttt{JavaScript}
\item \texttt{PHP}

\end{itemize}
\end{alertblock}
\end{frame}
}


\begin{frame}[label=Python]
\begin{block}{Python "---}

высокоуровневый язык программирования общего назначения, ориентированный на повышение производительности разработчика и читаемости кода. Синтаксис ядра Python минималистичен. В то же время стандартная библиотека включает большой объём полезных функций.
\end{block}
\end{frame}

\begin{frame}[shrink=10]
\begin{block}

Python поддерживает несколько парадигм программирования, в том числе структурное, объектно-ориентированное, функциональное, императивное и аспектно-ориентированное. Основные архитектурные черты — динамическая типизация, автоматическое управление памятью, полная интроспекция, механизм обработки исключений, поддержка многопоточных вычислений и удобные высокоуровневые структуры данных. Код в Python организовывается в функции и классы, которые могут объединяться в модули (они в свою очередь могут быть объединены в пакеты).
\end{block}
\end{frame}

\begin{frame}[shrink=10]
\begin{block}{}
Эталонной реализацией Python является интерпретатор CPython, поддерживающий большинство активно используемых платформ. Он распространяется под свободной лицензией Python Software Foundation License, позволяющей использовать его без ограничений в любых приложениях, включая проприетарные. Есть реализации интерпретаторов для JVM (с возможностью компиляции), MSIL (с возможностью компиляции), LLVM и других. Проект PyPy предлагает реализацию Python с использованием JIT"=компиляции, которая значительно увеличивает скорость выполнения Python"=программ.
\end{block}
\end{frame}

\begin{frame}
\begin{block}

Python — активно развивающийся язык программирования, новые версии (с добавлением/изменением языковых свойств) выходят примерно раз в два с половиной года. Вследствие этого и некоторых других причин на Python отсутствуют стандарт ANSI, ISO или другие официальные стандарты, их роль выполняет CPython.
\end{block}
\end{frame}

{\subsubsection{Пример кода на языке Python}

\begin{frame}[fragile]
\begin{block}{Пример кода на языке Python:}
\begin{lstlisting}
def fibonacci(max):        
    a, b = 0, 1
    while a < max:
        yield a            
        a, b = b, a + b    
for n in fibonacci(100):   
    print (n)
\end{lstlisting}
\end{block}

\begin{center}
\hyperlink{menu}{\beamerbutton{Ко всем языкам}}
\end{center}
\end{frame}
}






\begin{frame}[label=VisualBasic, shrink=10]
\begin{block}{Microsoft Visual Basic "---}
язык программирования, а также интегрированная среда разработки программного обеспечения, разрабатываемые корпорацией Microsoft. Язык Visual Basic унаследовал дух, стиль и отчасти синтаксис своего предка — языка BASIC, у которого есть немало диалектов. В то же время Visual Basic сочетает в себе процедуры и элементы объектно-ориентированных и компонентно-ориентированных языков программирования. 
Интегрированная среда разработки VB включает инструменты для визуального проектирования пользовательского интерфейса, редактор кода с возможностью IntelliSense и подсветкой синтаксиса, а также инструменты для отладки приложений.
\end{block}
\end{frame}

\begin{frame}
\begin{block}

Visual Basic также является хорошим средством быстрой разработки (RAD) приложений баз данных для операционных систем семейства Microsoft Windows. Множество готовых компонентов, поставляемых вместе со средой, призваны помочь программисту сразу же начать разрабатывать бизнес-логику приложения, не отвлекая его внимание на написание кода запуска программы, подписки на события и других механизмов, которые VB реализует автоматически.
\end{block}
\end{frame}

\begin{frame}
\begin{block}

Первое признание серьёзными разработчиками Visual Basic получил после выхода версии 3. Окончательное признание как полноценного средства программирования для Windows — при выходе версии 5. Версия VB6, входящая в состав Microsoft Visual Studio 6.0, стала по-настоящему зрелым и функционально богатым продуктом.
\end{block}
\end{frame}



\begin{frame}[fragile]{Пример кода на языке Visual Basic}
\begin{block}

\begin{lstlisting}
Private Declare Function mciExecute 
    Lib "winmm.dll"
    (ByVal lpstrCommand As String)
        As Long
Call mciExecute("play file")
Call mciExecute("close file")
\end{lstlisting}

\end{block}

\begin{center}
\hyperlink{menu}{\beamerbutton{Ко всем языкам}}
\end{center}




\end{frame}




\begin{frame}[label=Delphi]
\begin{block}{Delphi "---}
императивный структурированный объектно-ориентированный язык программирования со строгой статической типизацией переменных. Основная область использования — написание прикладного программного обеспечения.
\end{block}
\end{frame}
\begin{frame}[shrink=10]
\begin{block}{}

    Первоначально носил название Object Pascal и исторически восходит к одноимённому диалекту языка, разработанному в фирме Apple в 1986 году группой Ларри Теслера Однако в настоящее время термин Object Pascal чаще всего употребляется в значении языка среды программирования Delphi. Начиная с Delphi 7 в официальных документах компания Borland стала использовать название Delphi для обозначения языка Object Pascal.
\end{block}
\end{frame}
\begin{frame}
    \begin{block}
    
    Изначально среда разработки Delphi была предназначена исключительно для разработки приложений Microsoft Windows, затем был реализован вариант для платформ Linux (под торговой маркой Kylix), однако после выпуска в 2002 году Kylix 3 его разработка была прекращена, и вскоре было объявлено о поддержке Microsoft .NET, которая, в свою очередь, была прекращена с выходом Delphi 2007.
    \end{block}
\end{frame}

\begin{frame}
    \begin{block}
    
    В настоящее время, наряду с поддержкой разработки 32 и 64-разрядных программ для Windows, реализована возможность создавать приложения для Apple Mac OS X (начиная с Embarcadero Delphi XE2), iOS (включая симулятор, начиная с XE4 посредством собственного компилятора), Google Android (начиная с Delphi XE5), а также Linux Server x64 (начиная с версии 10.2 Tokyo).
    \end{block}
\end{frame}

\begin{frame}
    \begin{block}
    
    Независимая, сторонняя реализация среды разработки проектом Lazarus (Free Pascal, в случае компиляции в режиме совместимости с Delphi) позволяет использовать его для создания приложений на Delphi для таких платформ, как Linux, Mac OS X и Windows CE.

    Также предпринимались попытки использования языка в проектах GNU (например, Notepad GNU) и написания компилятора для GCC (GNU Pascal).
\end{block}
\end{frame}


\begin{frame}[fragile]
\begin{block}{Пример программы на Delphi}
\begin{lstlisting}
program Helloworld;        
{$APPTYPE CONSOLE}           
begin
  writeln('Hello, world!');    
  readln;                    
end.   
\end{lstlisting}

\end{block}
\begin{center}
\hyperlink{menu}{\beamerbutton{Ко всем языкам}}
\end{center}
\end{frame}



\begin{frame}[label=Perl,shrink=20]
\begin{block}{Perl "---}
высокоуровневый интерпретируемый динамический язык программирования общего назначения, созданный Ларри Уоллом, лингвистом по образованию. Название языка официально расшифровывается как Practical Extraction and Report Language. Первоначально название состояло из пяти символов и в таком виде в точности совпадало с английским словом pearl. Но затем стало известно, что такой язык существует, и букву «a» убрали. Символом языка Perl является верблюд — не слишком красивое, но очень выносливое животное, способное выполнять тяжёлую работу.
    Основной особенностью языка считаются его богатые возможности для работы с текстом, в том числе работа с регулярными выражениями, встроенная в синтаксис. Перл унаследовал много свойств от языков Си, AWK, скриптовых языков командных оболочек UNIX.
\end{block}
\end{frame}


\begin{frame}
\begin{block}

Perl — язык программирования общего назначения, который был первоначально создан для манипуляций с текстом, но на данный момент используется для выполнения широкого спектра задач, включая системное администрирование, веб-разработку, сетевое программирование, игры, биоинформатику, разработку графических пользовательских интерфейсов.
\end{block}
\end{frame}

\begin{frame}
\begin{block}

Язык можно охарактеризовать скорее как практичный (лёгкость в использовании, эффективность, полнота), чем красивый (элегантность, минималистичность) Главными достоинствами языка являются поддержка различных парадигм (процедурный, объектно-ориентированный и функциональный стили программирования), контроль за памятью (без сборщика мусора, основанного на циклах), встроенная поддержка обработки текста, а также большая коллекция модулей сторонних разработчиков.
\end{block}
\end{frame}

\begin{frame}
\begin{block}

Согласно Ларри Уоллу, у Perl есть два девиза. Первый — «Есть больше одного способа это сделать» («There’s more than one way to do it»), известный также под аббревиатурой TMTOWTDI. Второй слоган — «Простые вещи должны быть простыми, а сложные вещи должны быть возможными» («Easy things should be easy and hard things should be possible»).
\end{block}
\end{frame}

\begin{frame}[fragile]
\begin{block}{Пример программы на Perl}

\begin{lstlisting}
#!/usr/bin/perl

use warnings;

my $n=shift;
my ($a, $b)=(0, 1);

($a, $b)=($b, $a+$b) while $n--;
print "$a\n";
\end{lstlisting}
\end{block}

\begin{center}
\hyperlink{menu}{\beamerbutton{Ко всем языкам}}
\end{center}
\end{frame}


\begin{frame}[label=Ruby]
\begin{block}{Ruby "---}
динамический, рефлективный, интерпретируемый высокоуровневый язык программирования. Язык обладает независимой от операционной системы реализацией многопоточности, строгой динамической типизацией, сборщиком мусора и многими другими возможностями. По особенностям синтаксиса он близок к языкам Perl и Eiffel, по объектно-ориентированному подходу — к Smalltalk. Также некоторые черты языка взяты из Python, Lisp, Dylan и Клу.
\end{block}
\end{frame}



\begin{frame}
\begin{block}

Со времени выпуска публичной версии в 1995 году, Ruby привлек внимание программистов со всего мира. В 2006 году Ruby завоевал массовое признание. В крупнейших городах по всему миру активно действуют группы пользователей Ruby, а конференции, посвященные Ruby, заполнены до предела.
\end{block}
\end{frame}


\begin{frame}
\begin{block}

Ruby-Talk, основная рассылка для обсуждения языка Ruby, содержала в среднем 200 сообщений ежедневно в 2006 году. В последние годы это количество уменьшилось, так как сообщество стало использовать несколько тематических рассылок вместо одной общей.
\end{block}
\end{frame}

\begin{frame}
\begin{block}

Индекс TIOBE, который измеряет рост популярности языков программирования, помещает Ruby на 9 место среди общепризнанных языков программирования. Во многом, рост популярности языка приписывается популярности программного обеспечения, написанного на Ruby, в особенности – Ruby on Rails, среде разработки веб-приложений.
\end{block}
\end{frame}

\begin{frame}
\begin{block}

Ruby также абсолютно открыт. Открыт для бесплатного использования, изменений, копирования и распространения.
\end{block}
\end{frame}

\begin{frame}[fragile]
\begin{block}{Пример программы на Ruby}

\begin{lstlisting}
irb(main):002:0> puts "Hello World"
Hello World
=> nil
\end{lstlisting}
\end{block}
\end{frame}

\begin{frame}
И так далее оставшиеся языки
\end{frame}

\end{document}

