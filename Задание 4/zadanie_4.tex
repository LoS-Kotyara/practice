\documentclass[12pt,a4paper]{article}
\usepackage[utf8x]{inputenc}
\usepackage{ucs}
\usepackage[russian]{babel}
\usepackage[OT1]{fontenc}
\usepackage{amsmath}
\usepackage{amsfonts}
\usepackage{amssymb}
%\usepackage{matrix}
%\usepackage{pmatrix}
%\usepackage{bmatrix}
%\usepackage{vmatrix}
%\usepackage{Vmatrix}
\author{Давиденко Алексей}
\begin{document}
Рассмотрим систему линейных уравнений вида:\\
\par$\qquad
\left\{\begin{aligned}
& a_{11}x_1 + a_{12}x_2 + \ldots + a_{1n}x_n=b_1\\
& a_{21}x_1 + a_{22}x_2 + \ldots + a_{2n}x_n=b_2\\
&\ldots \ldots \ldots \ldots \ldots \ldots \ldots \ldots \ldots \ldots \\
& a_{m1}x_1 + a_{m2}x_2 + \ldots + a_{mn}x_n=b_m\\
\end{aligned}\right.\\
$

Эта система состоит из {\it m} линейных уравнений относительно {\it n} неизвестных. Она может быть записана в виде следующего матричного уравнения :
 \par $A_x = b$,\\
где

\par $A=
\left(\begin{matrix}
a_{11} & a_{12} & \ldots & a_{1n}\\
a_{21} & a_{22} & \ldots & a_{2n}\\
\vdots & \vdots & \ddots & \vdots\\
a_{m1} & a_{m2} & \ldots & a_{mn}\\
\end{matrix}\right);
$
$x=
\left(\begin{matrix}
x_1 \\
x_2 \\
\vdots \\
x_n \\
\end{matrix}\right)
$
$b=
\left(\begin{matrix}
b_1\\
b_2\\
\vdots\\
b_m\\
\end{matrix}\right)
$
\\
\par Матрица $A$ — это матрица коэффициентов системы линейных уравнений, вектор-столбец $x$  — вектор неизвестных, а вектор-столбец $b$ — некоторый заданный вектор.\\

\par Для того, чтобы система имела решение (хотя бы одно), необходимо и достаточно, чтобы вектор $b$ был линейной комбинацией столбцов $A$ , и тогда вектор $x$  — это вектор, содержащий коэффициенты разложения вектора $b$  по столбцам матрицы $A$ .

Матрица размера $m \times 1$ называется \textbf{вектор-столбцом} и имеет специальное обозначение:\\
\par $
colon(a_1, \ldots, a_i, \ldots, a_m) = 
\left(\begin{matrix}
a_1 \\
\vdots \\
a_i \\
\vdots \\
a_n \\
\end{matrix}\right)
= (a_1, \ldots, a_i, \ldots, a_m)^T
$
\\
\par Матрица размера $1 \times n$  называется вектор-строкой и имеет специальное обозначение:
\par $
row(a_1, \ldots, a_i, \ldots, a_n) = (a_1, \ldots, a_i, \ldots, a_n)
$

\newpage 
Если необходимо дать развёрнутое представление матрицы в виде таблицы, то используют запись вида
\[
\begin{pmatrix}
a_{11} & \ldots  & a_{1j} & \ldots & a_{1n} \\
\vdots & \ddots & \vdots & \ddots & \vdots \\
a_{i1} & \ldots  & a_{ij} & \ldots & a_{in} \\
\vdots & \ddots & \vdots & \ddots & \vdots \\
a_{m1} & \ldots  & a_{mj} & \ldots & a_{mn} \\
\end{pmatrix},~
\begin{bmatrix}
a_{11} & \ldots  & a_{1j} & \ldots & a_{1n} \\
\vdots & \ddots & \vdots & \ddots & \vdots \\
a_{i1} & \ldots  & a_{ij} & \ldots & a_{in} \\
\vdots & \ddots & \vdots & \ddots & \vdots \\
a_{m1} & \ldots  & a_{mj} & \ldots & a_{mn} \\
\end{bmatrix},~
\begin{Vmatrix}
a_{11} & \ldots  & a_{1j} & \ldots & a_{1n} \\
\vdots & \ddots & \vdots & \ddots & \vdots \\
a_{i1} & \ldots  & a_{ij} & \ldots & a_{in} \\
\vdots & \ddots & \vdots & \ddots & \vdots \\
a_{m1} & \ldots  & a_{mj} & \ldots & a_{mn} \\
\end{Vmatrix}
\]

\par Тело кватернионов $\mathbb{H}$ может быть (изоморфно) промоделировано над полем $\mathbb{R}$ вещественных чисел:

\begin{flushright}$
Q=
\left(\begin{matrix}
t & x & y & -z \\
-x & t & -z & -y \\
-y & z & t & x \\
z & y & -x & t \\
\end{matrix}\right)
$\end{flushright}

для $q=t+ix+jy+kz \in \mathbb{H}$  матричный аналог  , где $t,x,y,z \in \mathbb{R}$ .\\

\textbf{CKM-матрица, матрица Кабиббо— Кобаяши— Маскавы}
\\
\par\qquad
$ 
\begin{bmatrix}
V_{ud} & V_{us} & V_{ub} \\
V_{cd} & V_{cs} & V_{cb} \\
V_{td} & V_{ts} & V_{tb} \\
\end{bmatrix}
$
$
\begin{bmatrix}
|d\rangle \\
|s\rangle \\
|b\rangle \\
\end{bmatrix} = 
$
$
\begin{bmatrix}
|d'\rangle \\
|s'\rangle \\
|b'\rangle \\
\end{bmatrix} 
$
\\
\par Слева мы видим CKM-матрицу вместе с вектором сильных собственных состояний кварков, а справа имеем слабые собственные состояния кварков. ККМ-матрица описывает вероятность перехода от одного кварка $q$ к другому кварку $q'$ . Эта вероятность пропорциональна  $|V_{qq'}|^2$

\end{document}

