\documentclass{article}
\usepackage{ucs}
\usepackage[utf8x]{inputenc}
\usepackage[russian]{babel}
\usepackage{ulem}\normalem
\usepackage{color}
\usepackage{amssymb}
\begin{document}
{\bf Задание 2}

{\bfЛинейным отображением} векторного пространства $\mathrm{\it L}_K$ над полем {\Large \it K} в
векторное пространство $\mathrm{\it M}_K$ ({\bf линейным оператором} из $\mathrm{\it L}_K$ в $\mathrm{\it M}_K$) над тем
же полем называется отображение $\mathrm{f\colon {\it L}_K \to {\it M}_K}$,

удовлетворяющее {\it условию линейности}

$\mathrm{f(x+y)=f(x)+f(y)}$,

$\mathrm{f(\alpha x)=\alpha f(x)}$.

для всех $\mathrm{\it x,y \in {\it L}_K }$ и $\mathrm{\alpha \in {\Large \it K}}$ .

Если определить операции сложения и умножения на скаляр из основного поля {\Large \it K} как
\begin{itemize}
\item $\mathrm{(f+g)(x)=f(x)+g(x) \forall \in {\it L}_K}$.
\item $\mathrm{(kf)(x)=kf(x) \forall x \in {\it L}_K , \forall k \in {\Large \it K}}$
\end{itemize}

множество всех линейных отображений из $\mathrm{\it L}_K$ в $\mathrm{\it M}_K$ превращается в
векторное пространство, которое обычно обозначается как $\mathrm{\zeta({\it L}_K, {\it M}_K)}$


Если векторные пространства $\mathrm{\it L}_K$ и $\mathrm{\it M}_K$ являются линейными
топологическими пространствами, то есть на них определены топологии,
относительно которых операции этих пространств непрерывны, то можно
определить понятие ограниченного оператора: линейный оператор
называется ограниченным, если он переводит ограниченные множества в
ограниченные (в частности, все непрерывные операторы ограничены). В
частности, в нормированных пространствах множество ограничено, если
норма любого его элемента ограничена, следовательно, в этом случае
оператор называется ограниченным, если существует число N такое что $\mathrm{\forall x \in {\it L}_K,  {\|{\Large A} x \|}_{{\it M}_K} \le {{\Large N} \| x \|}_{{\it L}_K }}$. Можно показать, что в случае
нормированных пространств непрерывность и ограниченность операторов
эквивалентны. Наименьшая из постоянных N, удовлетворяющая указанному
выше условию, называется {\bf нормой оператора}:

$\mathrm{ \| {\Large A} \| = \sup\limits_{\|x\| \neq 0} \frac{\|{\Large A}_x\|}{\|x\|} = \sup\limits_{\|x\|=1} \|{\Large A}_x\|}$

Введение нормы операторов позволяет рассматривать пространство
линейных операторов как нормированное линейное пространство (можно
проверить выполнение соответствующих аксиом для введенной нормы). Если
пространство $\mathrm{\it M}_K$ — банахово, то и пространство линейных операторов тоже
банахово.

Оператор $\mathrm{{\Large A}^-1}$ называется обратным линейному оператору $\mathrm{\Large A}$ , если
выполняется соотношение:$\mathrm{{\Large A}^-1} {\Large A}={\Large A}{\Large A}^-1 = 1$

Оператор $\mathrm{{\Large A}^-1}$ , обратный линейному оператору $\mathrm{\Large A}$ , также является линейным
непрерывным оператором. В случае если линейный оператор действует из
банахового пространства в другое банахово пространство, то по теореме
Банаха обратный оператор существует.\par


{\bf Унитарный оператор} — оператор, область определения и область значений
которого — всё пространство, сохраняющий скалярное произведение $\mathrm({\Large A}x, {\Large A}y)=(x,y)$
, в частности, унитарный оператор сохраняет норму
любого вектора $\mathrm{ \|A\|=\sqrt{{\Large A},{\Large A}}=\sqrt{x,x}=\|x\|}$; оператор, обратный
унитарному, совпадает с сопряжённым оператором $\mathrm{{\Large A}^-1}$=$\mathrm{{\Large A}^{*}}$ ; норма
унитарного оператора равна 1; в случае вещественного поля К унитарный
оператор называют {\it ортогональным};
\end{document}