\documentclass{article}
\usepackage{ucs}
\usepackage[utf8x]{inputenc}
\usepackage[russian]{babel}
\usepackage{ulem}\normalem
\usepackage{color}
\usepackage{amssymb}
\begin{document}
{\bf Задание 3}

{\bf Связанные понятия:}
1. Образом подмножества $\mathrm { {\it{\Large M} \subset {\Large L)}_K} }$ относительно линейного
отображения A называется множество $\mathrm{}$ .
2. Ядром линейного отображения $\mathrm{}$ называется подмножество $\mathrm{ {\it {\Large AM}} = \{ {\Large A}_x : x \in {\Large M} \} }$
$\mathrm{{\Large A}}$, которое отображается в нуль:
$\mathrm{ \ker f = \{x \in {\Large A} | f(x) = 0 }$

Ядро линейного отображения образует подпространство в линейном
пространстве $\mathrm{\it Large A}$.

3. Образом линейного отображения $\mathrm{\it f}$ называется следующее
подмножество $\mathrm{\it Large B}$:
$\mathrm{ f = \{f(x) \in {\Large B} | x \in {\Large A}}$

Образ линейного отображения образует подпространство в
линейном пространстве $\mathrm{\Large B}$ .
4. Отображение $\mathrm{f : {\Large A} \times {\Large B} \rightarrow {\Large C}}$ прямого произведения линейных
пространств $\mathrm{\Large A}$ и $\mathrm{\Large B}$ в линейное пространство $\mathrm{\Large C}$ называется
билинейным, если оно линейно по обоим своим аргументам.
Отображение прямого произведения большего числа линейных
пространств $\mathrm{f : {\Large A}_1 \times ... \times {\Large A}_n \in {\Large B}}$ называется полилинейным, если
оно линейно по всем своим аргументам.

5. Оператор $\mathrm{\tilde {\Large L}  }$ называется линейным $\mathrm{\it неоднородным}$ (или $\mathrm{\it аффинным}$), если
он имеет вид
$\mathrm{\tilde {\Large L} = {\Large L} + \upsilon}$
где — $\mathrm{\it \Large L}$ линейный оператор, а $\mathrm{\upsilon}$ — вектор.
6. Пусть $\mathrm{{\Large A} : {\Large L}_K \rightarrow {\Large L}_K}$ . Подпространство $\mathrm{{\Large M} \subset {\Large L}_K}$ называется
$\mathrm{\it инвариантным}$ относительно линейного отображения, если $\mathrm{\forall x \in {\Large M}, {\Large A}x \in {\Large M}}$

.

Критерий инвариантности. Пусть $\mathrm{{\Large M} \subset {\Large L}_K}$ — подпространство,такое
что $\mathrm{\Large X}$ разлагается в прямую сумму: $\mathrm{}$ . Тогда $\mathrm{\Large M}$
инвариантно относительно линейного отображения $\mathrm{\Large A}$ тогда и только
тогда, когда $\mathrm{ {\Large P}_M {\Large AP}_M = {\Large AP}_M}$, где $\mathrm{{\Large P}_M}$ - проектор на
подпространство $\mathrm{\Large M}$ .
7. $\mathrm{\bf Фактор-операторы}$. Пусть $\mathrm{{\Large A} : {\Large L}_K \rightarrow {\Large L}_K}$ — линейный оператор и
пусть $\mathrm{\it \Large M}$ — некоторое инвариантное относительно этого оператора
подпространство. Образуем фактор-пространство $\mathrm{{\Large L}_K / {\sim}^M}$ по
подпространству $\mathrm{\Large M}$ . Тогда $\mathrm{\bf фактор-оператором}$ называется оператор $\mathrm {\Large A}^+$
действующий на $\mathrm{{\Large L}_K / {\sim}^M}$ по правилу: $\mathrm{ \forall x^+ \in {\Large L}_K / {\sim}^M}$, $\mathrm{{\Large A}^+ x^+ = [{\Large A}x]}$  ,
где $\mathrm{[{\Large A}x]}$ — класс из фактор-пространства, содержащий $\mathrm{{\Large A}x} $.
Примеры линейных однородных операторов:
\begin{itemize}
\item оператор дифференцирования: $\mathrm{ {\Large L} \{x(\dot)\} =y(t)=\frac{dx(t)}{dt} }$;
\item оператор интегрирования:
$\mathrm{ y(t)= \int\limits_{0}^{t} x(\tau) d \tau    }$
\item оператор умножения на определённую функцию
$\mathrm{ \varphi(t): y(t) = \varphi(t)x(t) }$;
\item оператор интегрирования с заданным «весом»
$\mathrm{ \varphi(t): y(t) = \int\limits_{0}^{t} x(\tau) \varphi(\tau) }$
\item оператор взятия значения функции $\mathrm{f}$ в конкретной точке $\mathrm{x_0}$:
$\mathrm{ {\Large L} \{f\} = f(x_0) }$
\item оператор умножения вектора на матрицу:
$\mathrm{ b={\Large A}x }$ ;
\item оператор поворота вектора.
\end{itemize}
Примеры линейных неоднородных операторов:
\begin{itemize}
\item Любое аффинное проеобразование;
\item $\mathrm{ y(t) = \frac{dx(t)}{dt} + \varphi(t) }$;
\item $\mathrm{ y(t) = \int\limits_{0}^{t} x(\tau)d\tau + \varphi(t) }$;
\item $\mathrm{ y(t) =  \varphi_1(t)x(t) = \varphi_2(t) }$;
\end{itemize}
где $\varphi$, $\varphi_1$, $\varphi_2$ — вполне определённые функции, а $\mathrm{x(t)}$—
преобразуемая оператором функция.

\end{document}