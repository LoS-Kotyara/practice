\documentclass{article}
\usepackage{ucs}
\usepackage[utf8x]{inputenc}
\usepackage[russian]{babel}
\usepackage{ulem}\normalem
\usepackage{color}
\usepackage{amssymb}
\usepackage {amsmath}
\begin{document}

{\bf Задание 4}

Рассмотрим систему линейных уравнений вида:
\\
\par$\qquad
\left\{\begin{aligned}
& a_{11}x_1 + a_{12}x_2 + \ldots + a_{1n}x_n=b_1\\
& a_{21}x_1 + a_{22}x_2 + \ldots + a_{2n}x_n=b_2\\
&\ldots \ldots \ldots \ldots \ldots \ldots \ldots \ldots \ldots \ldots \\
& a_{m1}x_1 + a_{m2}x_2 + \ldots + a_{mn}x_n=b_m\\
\end{aligned}\right .\\
$

Эта система состоит из линейных $\mathrm{\it m}$ уравнений относительно $\mathrm{\it n}$ неизвестных.
Она может быть записана в виде следующего матричного уравнения:
\\
$\mathrm{\it {\Large A} x = b}$ 
,

где $\mathrm{\it \Large A =}$ 
$\left(\begin{matrix}
a_{11} & a_{12} & \ldots & a_{1n}\\
a_{21} & a_{22} & \ldots & a_{2n}\\
\vdots & \vdots & \ddots & \vdots\\
a_{m1} & a_{m2} & \ldots & a_{mn}\\
\end{matrix}\right);
$
$x=
\left(\begin{matrix}
x_1 \\
x_2 \\
\vdots \\
x_n \\
\end{matrix}\right)
$
$b=
\left(\begin{matrix}
b_1\\
b_2\\
\vdots\\
b_m\\
\end{matrix}\right)
$


Матрица ${\it A}$ — это матрица коэффициентов системы линейных уравнений,
вектор-столбец ${\it x}$ — вектор неизвестных, а вектор-столбец ${\it b}$ — некоторый
заданный вектор.
Для того, чтобы система имела решение (хотя бы одно), необходимо и
достаточно, чтобы вектор ${\it b}$ был линейной комбинацией столбцов ${\it \Large A}$ , и тогда
вектор ${\it x}$ — это вектор, содержащий коэффициенты разложения вектора ${\it b}$ по
столбцам матрицы ${\it \Large A}$ .

Матрица размера $m \times 1$ называется {\bf вектор-столбцом} и имеет специальное
обозначение:


Матрица размера $1 \times n$ называется вектор-строкой и имеет специальное
обозначение:
\\
$colon(a_1, \ldots, a_i, \ldots, a_m) = 
\left(\begin{matrix}
a_1 \\
\vdots \\
a_i \\
\vdots \\
a_n \\
\end{matrix}\right)
= (a_1, \ldots, a_i, \ldots, a_m)^T
$
\newpage
\par Матрица размера $1 \times n$ называется {\bf вектор-строкой} и имеет специальное
обозначение: 
\\
$ row(a_1, \ldots, a_i, \ldots, a_n) = (a_1, \ldots, a_i, \ldots, a_n)$
\\
\par Если необходимо дать развёрнутое представление матрицы в виде таблицы,
то используют запись вида
\\
$
\begin{pmatrix}
a_{11} & \ldots  & a_{1j} & \ldots & a_{1n} \\
\vdots & \ddots & \vdots & \ddots & \vdots \\
a_{i1} & \ldots  & a_{ij} & \ldots & a_{in} \\
\vdots & \ddots & \vdots & \ddots & \vdots \\
a_{m1} & \ldots  & a_{mj} & \ldots & a_{mn} \\
\end{pmatrix},~
\begin{bmatrix}
a_{11} & \ldots  & a_{1j} & \ldots & a_{1n} \\
\vdots & \ddots & \vdots & \ddots & \vdots \\
a_{i1} & \ldots  & a_{ij} & \ldots & a_{in} \\
\vdots & \ddots & \vdots & \ddots & \vdots \\
a_{m1} & \ldots  & a_{mj} & \ldots & a_{mn} \\
\end{bmatrix},~
\begin{Vmatrix}
a_{11} & \ldots  & a_{1j} & \ldots & a_{1n} \\
\vdots & \ddots & \vdots & \ddots & \vdots \\
a_{i1} & \ldots  & a_{ij} & \ldots & a_{in} \\
\vdots & \ddots & \vdots & \ddots & \vdots \\
a_{m1} & \ldots  & a_{mj} & \ldots & a_{mn} \\
\end{Vmatrix}
$
Тело кватернионов $\mathbb{H}$ может быть (изоморфно) промоделировано над полем $\mathbb{R}$
вещественных чисел:
\\

\begin{flushright}$
Q=
\left(\begin{matrix}
t & x & y & -z \\
-x & t & -z & -y \\
-y & z & t & x \\
z & y & -x & t \\
\end{matrix}\right)
$\end{flushright}

для $q=t+ix+jy+kz \in \mathbb{H}$ матричный аналог ,
где $t,x,y,z \in \mathbb{R}$.
\\

{\bf CKM-матрица, матрица Кабиббо—Кобаяши—Маскавы}
\\
$ 
\begin{bmatrix}
V_{ud} & V_{us} & V_{ub} \\
V_{cd} & V_{cs} & V_{cb} \\
V_{td} & V_{ts} & V_{tb} \\
\end{bmatrix}
$
$
\begin{bmatrix}
|d\rangle \\
|s\rangle \\
|b\rangle \\
\end{bmatrix} = 
$
$
\begin{bmatrix}
|d'\rangle \\
|s'\rangle \\
|b'\rangle \\
\end{bmatrix} 
$
\\
\\
Слева мы видим CKM-матрицу вместе с вектором сильных собственных
состояний кварков, а справа имеем слабые собственные состояния кварков.
ККМ-матрица описывает вероятность перехода от одного кварка q к другому
кварку ${\it q}^`$; . Эта вероятность пропорциональна
 $|V_{qq'}|^2$


\end{document}